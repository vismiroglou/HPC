\documentclass{article}
\begin{document}
\section*{Lecture 1 - Computer Architectures}
\indent Modern computers follow the Von Neumann architecture with \textit{fetch, execute, store} control flow. 
More specficially, the computer contains
\begin{itemize}
    \item A processing unit with both an arithmetic logic unit  and processor registers
    \item A control unit that includes an instruction register and a program counter
    \item Memory that stores data and instructions
    \item External mass storage
    \item Input and output mechanisms
\end{itemize}
The control unit manages the four basic operations as follows:
\begin{itemize}
    \item Fetch: gets the next program command from the computer's memory
    \item Decode: deciphers what the program is telling the computer to do.
    \item Execute: carries out the requested action
    \item Store: saves the result to a register or memory
\end{itemize}
The register is a type of memory that is very fast and very close to the cpu but very limited in space. It is used to save intermediate calculations. 
\end{document}